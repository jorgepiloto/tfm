\chapter*{Abstract}

The discovery of interstellar objects opens a new window to study asteroids and
comets formed in other planetary systems. Their highly eccentric orbits
challenge our ability to reach these bodies, but the giant scientific return
well requires all our efforts. In this work a deep dynamic analysis on the most
suitable transfer orbits to the two first identified interstellar interlopers:
1I/'Oumuamua and 2I/Borisov is presented. The analysis exemplifies the intrinsic
difficulty behind the development of dedicated missions to intercept these
bodies. By analyzing different launch scenarios, porkchop plots for the specific
energy at launch are generated for both prograde and retrograde transfers. In
addition, isolines for the total time of flight and the velocity at arrival are
computed for a big span of launch and arrival dates. The entire study allows for
the finding of the lowest energy transfer orbits, key to plan and develop future
missions to other discovered extrasolar objects. Different launch possibilities
are explored: from Earth, Lagrange point L2, or other gravity assisted
maneuvers. To catch a future visitor a preferred scenario arises: using the L2
point to quickly release a rendezvous mission able to study it during close
approach. The results obtained in this work are consistent with those found by
other proposed missions, like the Comet Interceptor, showing that L2 is a good
starting point for launching missions to intercept interstellar visitors. By
parking a spacecraft at this point, more reaction time is gained to plan an
optimum transfer trajectory. This is evinced by the fact that the optimum
transfer takes place before the interlopers were discovered. As a direct outcome
of this work the most favourable approach is a direct transfer between this
Lagrange point and the discovered interstellar interloper unless ideal
conditions for a gravity-assisted maneuver are met.

\vspace{4cm}
\textbf{Keywords:} Interstellar interlopers, transfer orbits, porkchop plots, Lagrange points,
gravity-assisted maneuvers, interstellar missions, surveillance research,
mission design, celestial exploration
