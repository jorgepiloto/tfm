\chapter*{Abstract}

This research offers a thorough examination of transfer orbits specifically
tailored for intercepting interstellar visitors. Through an exhaustive analysis
of diverse launch scenarios, detailed porkchop plots are crafted to illustrate
the specific energy required at launch for both prograde and retrograde
transfers. Additionally, isolines delineating the total time of flight and
velocity upon arrival are meticulously calculated across a broad spectrum of
launch and arrival dates. These efforts enable the precise determination of
optimal energy transfer paths. Departure strategies from Earth, Lagrange point
L2, and gravity-assisted maneuvers are thoroughly explored, providing insights
into their efficacy.

The study's outcomes harmonize with established mission proposals, exemplified
by the Comet Interceptor, corroborating the strategic significance of Lagrange
point L2 as a launch locus for interstellar missions. This underscores the
imperative for enhancing surveillance research initiatives centered on
identifying interlopers and fortifying pre-planned missions stationed at
Lagrange point L2. Given the validated efficacy of L2 as a launch platform,
heightened emphasis on surveillance research emerges as essential, ensuring
preparedness to capitalize on evolving opportunities for celestial exploration
and scientific inquiry.


\vspace{4cm}
\textbf{Keywords:} Interstellar interlopers, transfer orbits, porkchop plots, Lagrange points,
gravity-assisted maneuvers, interstellar missions, surveillance research,
mission design, celestial exploration
