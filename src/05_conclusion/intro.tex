\chapter{Conclusions}

After the direct transfer analysis from Earth in \ref{ch:direct-transfer}, the
direct transfer analysis from L2 in \ref{sec:lagrange-points-analysis}, and a
the gravity assist review in \ref{sec:gravity-assist-analysis}, this chapter
sumarizes all the results and presents the final conclusions.

\section{Optimum direct transfer: Earth vs L2}

Results for the optimum direct transfer computed previous chapters are
summarized in tables \ref{tab:summary-results-v-launch},
\ref{tab:summary-results-c3-launch} and \ref{tab:summary-results-arrival-v}. A
high reduction in the launch energy is observed when launching from L2 instead
of Earth. This reduction is mainly due to the low escape velocity at L2, which
allows for a more efficient transfer.

\vspace{1cm}
\begin{table}[H]
  \centering
  \begin{tabular}{|c|c|c|c|}
    \hline
    Object       & $\Delta v$ launch Earth [km/s] & $\Delta v$ launch L2 [km/s] & Reduction [\%] \\
    \hline
    1I/'Oumuamua & 13.85                          & 3.80                        & 72.56          \\
    \hline
    2I/Borisov   & 16.90                          & 5.85
                 & 65.38                                                                         \\
    \hline
  \end{tabular}
  \caption[Comparison of the launch velocity for direct transfers from Earth and
    L2.]{Comparison of the launch velocity for direct transfers from Earth and
    L2. These values are feasible considering modern propulsion technology,
    see \cite{longhurst2021}.}
  \label{tab:summary-results-v-launch}
\end{table}

\vspace{1cm}
\begin{table}[H]
  \centering
  \begin{tabular}{|c|c|c|c|}
    \hline
    Object       & $C_3$ launch Earth [km$^2$/s$^2$] & $C_3$ launch L2 [km$^2$/s$^2$] & Reduction [\%] \\
    \hline
    1I/'Oumuamua & 192.00                            & 14.41                          & 92.51          \\
    \hline
    2I/Borisov   & 286.00                            & 34.30                          & 88.08          \\
    \hline
  \end{tabular}
  \caption[Comparison of the launch energy for direct transfers from Earth and
    L2.]{Comparison of the launch energy for direct transfers from Earth and L2. A reduction in energy is observed when launching from L2.}
  \label{tab:summary-results-c3-launch}
\end{table}

\vspace{1cm}
\begin{table}[H]
  \centering
  \begin{tabular}{|c|c|c|c|}
    \hline
    Object       & $\Delta V$ arrival Earth [km/s] & $\Delta V$ arrival L2 [km/s] & Reduction [\%] \\
    \hline
    1I/'Oumuamua & 62.33                           & 61.46                        & 1.40           \\
    \hline
    2I/Borisov   & 33.00                           & 33.02                        & -0.06          \\
    \hline
  \end{tabular}
  \caption[Comparison of the arrival velocity for direct transfers from Earth and
    L2.]{Comparison of the arrival velocity for direct transfers from Earth and
    L2. Due to the close proximity of L2 to the Earth, the reduction in
    arrival speed is low compared to the reduction in launch energy.}
  \label{tab:summary-results-arrival-v}
\end{table}




Note that the required $\Delta v$ values for a transfer originating at L2 can be
achieved with modern propulsion technology and gravitational assists. However,
this research considers the usage of impulsive maneuvers, that is, maneuvers
that occur instantaneously. In reality, the spacecraft would need to perform
a maneuver over a period of time. Not only this, figure \ref{fig:payload_vs_c3}
only considers main space launchers, which are used for launching satellites.

For a spacecraft placed at L2, a bipropellant chemical propulsion system, such
as one using liquid oxygen and liquid hydrogen or liquid oxygen and liquid
ethanol, would be a feasible option to provide the required 3.5 km/s of $\Delta
  v$. The high specific impulse of bipropellant systems makes them capable of
delivering this level of performance.

Leveraging the advantageous position of L2 not only facilitates spacecraft
parking while awaiting the discovery of new ISOs but also augments mission
adaptability. By pre-positioning a spacecraft in space, it affords greater
maneuverability and flexibility in mission planning. This positioning grants
extended reaction time, enabling meticulous mission strategizing and the
optimization of trajectory paths.

Tables \ref{tab:optimum-launch-dates-oumuamua} and
\ref{tab:optimum-launch-dates-borisov} summary the optimum launch dates for
1I/'Oumuamua and 2I/Borisov, respectively. It is worth mentioning that the
optimum launch dates for 1I/'Oumuamua and 2I/Borisov take place before humanity
discovered them. This fact highlights the importance of increasing the
investment in space surveillance and tracking systems. By doing so, the reaction
time to intercept an ISO could be reduced significantly, allowing for more
efficient missions.

\vspace{1cm}
\begin{table}[H]
  \centering
  \begin{tabular}{|c|c|c|c|}
    \hline
    Object       & Optimum date launch Earth & Optimum date launch L2 & Discovery date \\
    \hline
    1I/'Oumuamua & 2017-01-20                & 2017-01-22             & 2017-10-19     \\
    \hline
  \end{tabular}
  \caption[Optimum launch dates for 1I/'Oumuamua compared to discovery its
    discovery date]{Optimum launch dates for
    1I/'Oumuamua compared to its discovery date. The first interstellar
    interloper was discovered close to 270 days after an optimum transfer
    could have been launched from Earth or L2.}
  \label{tab:optimum-launch-dates-oumuamua}
\end{table}

\vspace{1cm}
\begin{table}[H]
  \centering
  \begin{tabular}{|c|c|c|c|}
    \hline
    Object     & Optimum date launch Earth & Optimum date launch L2 & Discovery date \\
    \hline
    2I/Borisov & 2018-07-12                & 2018-07-12             & 2019-08-30     \\
    \hline
  \end{tabular}
  \caption[Optimum launch dates for 2I/Borisov compared to discovery its
    discovery date]{Optimum launch dates for
    2I/Borisov compared to its discovery date. The second interstellar
    interloper was discovered close to 414 days after an optimum transfer
    could have been launched from Earth or L2.}
  \label{tab:optimum-launch-dates-borisov}
\end{table}

\section{About gravity assists}

Regarding the analysis of gravity assists, it is determined that their
feasibility hinges on several factors. These encompass the relative spatial
configurations of planets during the launch phase, the duration of the mission,
the inclination concerning the ecliptic plane of the ISO, the velocity of the
interloper, and the launch velocity itself. In scenarios ripe with optimism, the
feasibility of executing multiple gravity assists, possibly involving inner
planets, emerges as a promising prospect.

These results mirror the strategic approach undertaken by the Comet Interceptor
mission. The mission advocates for the adoption of a direct external transfer
orbit from L2 in a ready-to-launch state to the interloper, presenting it as an
initial solution to surmount the intricacies associated with interplanetary
transfer challenges.

\section{Future work}

While working on this project, the author noticed a lack of robust software for
optimizing gravity assist trajectories. Trajectory optimization in astrodynamics
is complex, especially when dealing with multiple planetary flybys and impulsive
maneuvers.

Creating software to optimize gravity assist trajectories considering variables
like planetary configurations, time spans, and fuel constraints would be highly
beneficial. Such software could quickly design an optimal mission to intercept
an interstellar visitor, helping astrodynamicists overcome challenges like high
relative speeds and inclinations.
