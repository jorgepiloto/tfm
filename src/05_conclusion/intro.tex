\chapter{Conclusion}

After the direct transfer analysis from Earth in \ref{ch:direct-transfer}, the
direct transfer analysis from L2 in \ref{sec:lagrange-points-analysis}, and a
the gravity assist review in \ref{sec:gravity-assist-analysis}, this chapter
sumarizes all the results and presents the final conclusions.

\section{Summary of results}

Results collected in the previous chapters are summarized in table
\ref{tab:summary-results-c3}. A high reduction in the launch energy is observed
when launching from L2 instead of Earth. This reduction is mainly due to the low
escape velocity at L2, which allows for a more efficient transfer.

\vspace{1cm}
\begin{table}[H]
  \centering
  \begin{tabular}{|c|c|c|c|}
    \hline
    Object       & $C_3$ launch Earth [km$^2$/s$^2$] & $C_3$ launch L2 [km$^2$/s$^2$] & Reduction [\%] \\
    \hline
    1I/'Oumuamua & 192.00                            & 14.41                          & 92.51          \\
    \hline
    2I/Borisov   & 286.00                            & 34.30                          & 88.08          \\
    \hline
  \end{tabular}
  \caption[Comparison of the launch energy for direct transfers from Earth and
    L2.]{Comparison of the launch energy for direct transfers from Earth and L2.}
  \label{tab:summary-results-c3}
\end{table}

Furthermore, L2 allows for parking a spacecraft while waiting for new ISOs to be
discovered. By having a spacecraft already in space, the flexibility of the
mission is increased. More reaction time is available to plan the mission and
optimize the trajectory.

From the gravity assist analysis, it is concluded that the use of a gravity
assist depends on various factors including:

\begin{itemize}
  \item The relative position of the planets at the time of the launch
  \item The time of flight of the mission
  \item The inclination with respect to the ecliptic plane of the ISO
  \item The velocity of the interloper
  \item The launch velocity
\end{itemize}

It is likely that a multiple gravity assist with some of the inner planets could
be performed in an optimistic scenario.

This conclusion is similar to the one achieved by the Comet Interceptor mission,
where a direct extern transfer orbit from L2 to the interloper is proposed as a
first solution to the transfer problem.

\section{Future work}

While working on this work, the author has found a lack of robust software for
optimizing gravity assist trajectories.

Trajectory optimization in the context of astrodynamics is a complex topic.
Gravity assist can also become complicated if considering multiple flybys on
multiple planets, and impulsive maneuvers in between.

The development of a software tool for optimizing gravity assist trajectories
given a set of planets, times spans, fuel constrains, and other requirements
would be a great tool.

If such a software is devised, it could be used to quickly design an optimum
mission to intercept an ISO. By doing so, astrodynamicists overcome the high
relative speeds and inclinations expected to be found in interlopers.

Finally, the simulation of synthetic ISOs could be used to test the software and
prepare draft mission flight plans, reducing even more the reaction time. If
combined with the optimization software, the achieved results could be
outstanding.
