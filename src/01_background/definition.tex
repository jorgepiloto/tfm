\section{Definition, origin and attributes}

Interstellar objects (ISOs) are asteroids, comets or planets that are not
gravitationally bound to a star, despite having spurious encounters with these
last objects in their path trough the interstellar medium (ISM).

Their origin can vary from being ejected from their parent star, to being
accreted by a planet and later ejected, to being formed in the interstellar
a medium.

As the solar system moves within the Local Interstellar Cloud (LIC), also
referred to as the Local Fluff, interstellar materials penetrates the
heliosphere\footnote{The heliosphere is the region of space dominated by the
Sun's influence. It extends beyond the orbit of Pluto and presents a drop shape
with its tail in the opposite direction of motion of the solar system through
the LIC.} at a rate of $26\text{km/s}$ \cite{hajdukova2020}. These interstellar
materials present a variety of sizes ranging from a few micrometers up to a few
hundreds of meters. Depending on their size and composition, they are subjected
to different forces, such as radiation pressure, solar wind, and the
gravitational pull of the Sun and planets.

\cite{sterken2012} defines the parameter $\beta$ as the ratio between radiation
force and gravity force:

\begin{equation}
    \beta = \frac{|\bm{F_{\text{rad}}}|}{|\bm{F_{\text{G}}}|} = \frac{A_p Q_{pr} S_0}{cG M_{0} m_p}
    \label{eq:deflection_parameter}
\end{equation}

Equation \ref{eq:deflection_parameter} shows that $\beta$ is a function of the
geometric albedo of the particle $A_p$, the radiation pressure coefficient $Q_{pr}$,
the solar constant $S_0$, the speed of light $c$, the gravitational constant $G$,
the solar mass $M_0$, and the particle mass $m_p$.

Three different regimes can be identified according to the value of $\beta$:

\begin{itemize}
    \item $\beta \gg 1$: Radiation pressure dominates over gravity. In this case,
          the particle follows a hyperbolic orbit away from the Sun.
    \item $\beta \sim 1$: Radiation pressure and gravity are comparable. In this
          case, the particle follows a rectilinear orbit towards the sun. Note
          that this is an ideal scenario and small perturbations in both
          forces can lead to a non-linear trajectory.
    \item $\beta \ll 1$: Gravity dominates over radiation pressure. In this
          scenario, gravity dominates and the particle falls towards the Sun in an
          hyperbolic orbit.
\end{itemize}

% TODO: attach figure to show deflection of particles

The heliosphere acts as a shield that protects the solar system from the ISM. It
can deflect small particles. However, larger ISOs can bypass the heliosphere
and enter the solar system.

Discretizing whether a body is an ISO is a challenging task. This is
particularly true for small objects, which are harder to detect and track,
leading to larger measurement errors. 

The following attributes can indicate the interstellar nature of an object:

\begin{itemize}
    \item \textbf{Hyperbolic orbit}: ISOs present hyperbolic orbits since they
          are not gravitationally bound to the Sun. This translates into
          eccentricities greater than the unity.
    \item \textbf{High relative velocity}: ISOs present high relative velocities
          with the planets and other solar system bodies. This is a consequence
          of their interstellar origin.
    \item \textbf{High inclination}: The plane of the solar system is well
          defined by the planets and other bodies. Thus, a body with a high
          inclination and an hyperbolic orbit is likely to be have an
          interstellar origin.
\end{itemize}

Note that these attributes are not exclusive to ISOs. For example, a body with a
high inclination and a hyperbolic orbit could be a comet from the Oort
cloud\footnote{ Oort cloud is a trans-Neptunian region that extends from 2000 to
50000 AU. It is the source of long-period comets and believed to contain a total
mass of 5 Earth masses made up to $10^{12}$ - $10^{14}$ objects. }. However,
studies \cite{francis2005} show that it is more likely that comets originated in the Oort cloud are
ejected into the ISM. Estimations indicate that the ratio between expelled and
retained comets ranges between $\eta = \frac{\text{Comets
expelled}}{\text{Comets retained}} = 3 \text{ - } 100$.
