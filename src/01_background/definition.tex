\section{Fundamentals}

This section provides a brief overview of interstellar objects. For a deeper
review on interstellar objects, the reader is referred to \cite{jewitt2022},
whose contribution to the topic is invaluable.

\subsection{Definition}
Interstellar objects (ISOs) are asteroids, comets or planetary bodies moving
through interstellar medium (ISM) without being gravitationally bound to a star.
Eventually, ISOs can pass through a planetary system, such as the solar system.
Some of them may be even captured, as suggested by \cite{napier2021}.

ISOs are also referred to as interstellar interlopers \cite{jewitt2022}, as they
can be seen as intruders travelling through a different system from their
original one.

\subsection{Origin}
There are different mechanisms that can lead to the ejection of ISOs from their
original system. The most common are:

\begin{itemize}
  \item \textbf{Stellar encounters.} ISOs can be ejected from their original
        system due to gravitational interactions with other stars. This
        mechanism is particularly relevant in dense stellar environments, such
        as globular clusters, see \cite{fouchard2011}.

  \item \textbf{Planetary encounters.} Planetary encounters can also lead to
        the ejection of ISOs. This mechanism is particularly relevant in
        planetary systems with large planets, such as Jupiter and Saturn, see
        \cite{horner2003}.

  \item \textbf{Stellar explosions.} Supernovae and other stellar explosions
        can also lead to the ejection of ISOs. These events can provide the
        necessary energy to eject ISOs from their original system, see
        \cite{portegies2018}.
\end{itemize}

\subsection{Abundance}
The abundance of ISOs in the galaxy is still an open question due to the lack of
enough data to make a reliable estimation. This lack of data has lead
researchers to generate synthetic populations of ISOs to estimate density
limits. Table \ref{tab:iso_density_limits} shows the density limits estimated
by different studies:

\begin{table}[H]
  \centering
  \begin{tabular}{|c|c|c|}
    \hline
    \textbf{Study}       & \textbf{Density limit ($1/\text{pc}^3$)} & \textbf{Density limit ($1/\text{AU}^3$)} \\
    \hline
    \cite{gaidos2017}    & $1.0 \times 10^{14}$                     & $1.1 \times 10^{-02}$                    \\
    \cite{jewitt2017}    & $8.0 \times 10^{14}$                     & $9.1 \times 10^{-02}$                    \\
    \cite{portegies2018} & $1.0 \times 10^{14}$                     & $1.1 \times 10^{-02}$                    \\
    \cite{feng2018}      & $4.8 \times 10^{13}$                     & $5.5 \times 10^{-03}$                    \\
    \cite{fraser2018}    & $8.0 \times 10^{14}$                     & $9.1 \times 10^{-02}$                    \\
    \cite{do2018}        & $2.0 \times 10^{15}$                     & $2.3 \times 10^{-01}$                    \\
    \hline
  \end{tabular}
  \caption[Estimated abundance of interstellar interlopers]{Density limits estimated by different studies. Note that future
    discoveries and improvements in detection techniques can lead to different
    estimations. Adapted from the original review of \cite{moro2023}.}
  \label{tab:iso_density_limits}
\end{table}


\subsection{Expected orbit attributes}
\label{sec:expected_orbit_attributes}

Discretizing whether a body is an ISO is a challenging task. This is
particularly true for small objects, which are harder to detect and track,
leading to larger measurement errors when defining their orbits.

The following attributes can indicate the interstellar nature of an object:

\begin{itemize}
  \item \textbf{Hyperbolic orbit.} ISOs must present hyperbolic orbits since
        they are not gravitationally bound to the Sun. This translates into
        eccentricities greater than the unity. Hyperbolic eccentricities have
        been identified in our solar system but attributed to the gravitational
        pull of the outter planets.
  \item \textbf{High relative velocity.} Interlopers present high relative
        velocities with the planets and other bodies of the system they are
        passing through. This is a consequence of their interstellar origin.
  \item \textbf{High inclination with respect to invariable plane of the solar
          system.} Although ISOs are expected to be discovered with any
        inclination, a high inclination with respect to the invariable plane of
        the solar system can be an indication of their interstellar origin.
        Since an ISO is not gravitationally bound to the Sun, it is unlikely
        that its angular momentum is aligned with the one of the solar system.
\end{itemize}

\subsection{Interstellar impostors}

Note that the attributes presented in subsection
\ref{sec:expected_orbit_attributes} are not exclusive to ISOs. For example, a
body with a high inclination and a hyperbolic orbit could be a comet from the
Oort cloud\footnote{ Oort cloud is a trans-Neptunian region that extends from
  2000 to 50000 AU. It is the source of long-period comets and believed to contain
  a total mass of 5 Earth masses made up to $10^{12}$ - $10^{14}$ objects. }
affected by the gravitational pull of the outter planets.

These objects are known as \textit{interstellar impostors}, as they exhibit
their properties but not their origin. Other authors like \cite{higuchi2020}
refer to these objects as hyperbolic Oort cloud comets (HOCs). HOCs are more
likely to be ejected into the ISM rather than falling into the internals of the
solar system, see \cite{francis2005}. However, \cite{eloy2024} have found an
Earth impactor with a hyperbolic orbit, which is believed to be an HOC resulting
from the perturbation of Oort's cloud.
