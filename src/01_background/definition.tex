\section{Definition, origin, abundance, and other attributes}

This section provides a brief overview of interstellar objects. For a deeper
review on interstellar objects, the reader is referred to \cite{jewitt2023},
whose 

% Definition
Interstellar objects (ISOs) are asteroids, comets planets moving through
interstellar medium (ISM) without being gravitationally bound to a star.
Eventually, ISOs can pass through a planetary system, such as the solar system.
Some of them may be even captured.

ISOs are also referred to as interstellar interlopers \cite{jewitt2023}, as they
can be seen as intruders travelling through a different system from their
original one.

% Origin
There are different mechanisms that can lead to the ejection of ISOs from their
original system. The most common are:

\begin{itemize}
    \item \textbf{Stellar encounters}: ISOs can be ejected from their original
          system due to gravitational interactions with other stars. This
          mechanism is particularly relevant in dense stellar environments, such
          as globular clusters, see \cite{portegies2018}.
            
    \item \textbf{Planetary encounters}: Planetary encounters can also lead to
        the ejection of ISOs. This mechanism is particularly relevant in
        planetary systems with large planets, such as Jupiter and Saturn, see
        \cite{kaib2011}.
          
    \item \textbf{Stellar explosions}: Supernovae and other stellar explosions
        can also lead to the ejection of ISOs. These events can provide the
        necessary energy to eject ISOs from their original system, see
        \cite{portegies2018}.
\end{itemize}

% Abundance
The abundance of ISOs in the galaxy is still an open question due to the lack of
enough data to make a reliable estimation. This lack of data has lead
researchers to generate synthetic populations of ISOs to estimate density
limits. Table \ref{tab:iso_density_limits} shows the density limits estimated
by different studies:

% Table with the following values
% Gaidos 2017: 1.0e+14 1 / pc3 = 1.1e-02 1 / AU3
% Jewitt 2017: 8.0e+14 1 / pc3 = 9.1e-02 1 / AU3
% Portegies 2018: 1.0e+14 1 / pc3 = 1.1e-02 1 / AU3
% Feng 2018: 4.8e+13 1 / pc3 = 5.5e-03 1 / AU3
% Fraser 2018: 8.0e+14 1 / pc3 = 9.1e-02 1 / AU3
% Do 2018: 2.0e+15 1 / pc3 = 2.3e-01 1 / AU3

\begin{table}[H]
    \centering
    \begin{tabular}{|c|c|c|}
        \hline
        \textbf{Study} & \textbf{Density limit ($1/\text{pc}^3$)} & \textbf{Density limit ($1/\text{AU}^3$)} \\
        \hline
        \cite{gaidos2017} & $1.0 \times 10^{14}$ & $1.1 \times 10^{-02}$ \\
        \cite{jewitt2017} & $8.0 \times 10^{14}$ & $9.1 \times 10^{-02}$ \\
        \cite{portegies2018} & $1.0 \times 10^{14}$ & $1.1 \times 10^{-02}$ \\
        \cite{feng2018} & $4.8 \times 10^{13}$ & $5.5 \times 10^{-03}$ \\
        \cite{fraser2018} & $8.0 \times 10^{14}$ & $9.1 \times 10^{-02}$ \\
        \cite{do2018} & $2.0 \times 10^{15}$ & $2.3 \times 10^{-01}$ \\
        \hline
    \end{tabular}
    \caption{Density limits estimated by different studies. Note that future
    discoveries and improvements in detection techniques can lead to different
    estimations. Adapted from \cite{moro2023}.}
    \label{tab:iso_density_limits}
\end{table}






% ARTILCES
% file:///home/jorge/Downloads/Moro-Mart%C3%ADn_2009_ApJ_704_733.pdf
% file:///home/jorge/Downloads/Interstellar_Planetesimals_Potential_Seeds_for_Pla.pdf
% Synthetic population: https://arxiv.org/pdf/2301.09375.pdf





As the solar system moves within the Local Interstellar Cloud (LIC), also
referred to as the Local Fluff, interstellar materials penetrates the
heliosphere\footnote{The heliosphere is the region of space dominated by the
Sun's influence. It extends beyond the orbit of Pluto and presents a drop shape
with its tail in the opposite direction of motion of the solar system through
the LIC.} at a rate of $26\text{km/s}$ \cite{hajdukova2020}. These interstellar
materials present a variety of sizes ranging from a few micrometers up to a few
hundreds of meters. Depending on their size and composition, they are subjected
to different forces, such as radiation pressure, solar wind, and the
gravitational pull of the Sun and planets.

\cite{sterken2012} defines the parameter $\beta$ as the ratio between radiation
force and gravity force:

\begin{equation}
    \beta = \frac{|\bm{F_{\text{rad}}}|}{|\bm{F_{\text{G}}}|} = \frac{A_p Q_{pr} S_0}{cG M_{0} m_p}
    \label{eq:deflection_parameter}
\end{equation}

Equation \ref{eq:deflection_parameter} shows that $\beta$ is a function of the
geometric albedo of the particle $A_p$, the radiation pressure coefficient $Q_{pr}$,
the solar constant $S_0$, the speed of light $c$, the gravitational constant $G$,
the solar mass $M_0$, and the particle mass $m_p$.

Three different regimes can be identified according to the value of $\beta$:

\begin{itemize}
    \item $\beta \gg 1$: Radiation pressure dominates over gravity. In this case,
          the particle follows a hyperbolic orbit away from the Sun.
    \item $\beta \sim 1$: Radiation pressure and gravity are comparable. In this
          case, the particle follows a rectilinear orbit towards the sun. Note
          that this is an ideal scenario and small perturbations in both
          forces can lead to a non-linear trajectory.
    \item $\beta \ll 1$: Gravity dominates over radiation pressure. In this
          scenario, gravity dominates and the particle falls towards the Sun in an
          hyperbolic orbit.
\end{itemize}

% TODO: attach figure to show deflection of particles

The heliosphere acts as a shield that protects the solar system from the ISM. It
can deflect small particles. However, larger ISOs can bypass the heliosphere
and enter the solar system.

Discretizing whether a body is an ISO is a challenging task. This is
particularly true for small objects, which are harder to detect and track,
leading to larger measurement errors. 

The following attributes can indicate the interstellar nature of an object:

\begin{itemize}
    \item \textbf{Hyperbolic orbit}: ISOs present hyperbolic orbits since they
          are not gravitationally bound to the Sun. This translates into
          eccentricities greater than the unity.
    \item \textbf{High relative velocity}: ISOs present high relative velocities
          with the planets and other solar system bodies. This is a consequence
          of their interstellar origin.
    \item \textbf{High inclination}: The plane of the solar system is well
          defined by the planets and other bodies. Thus, a body with a high
          inclination and an hyperbolic orbit is likely to be have an
          interstellar origin.
\end{itemize}

Note that these attributes are not exclusive to ISOs. For example, a body with a
high inclination and a hyperbolic orbit could be a comet from the Oort
cloud\footnote{ Oort cloud is a trans-Neptunian region that extends from 2000 to
50000 AU. It is the source of long-period comets and believed to contain a total
mass of 5 Earth masses made up to $10^{12}$ - $10^{14}$ objects. }. However,
studies \cite{francis2005} show that it is more likely that comets originated in the Oort cloud are
ejected into the ISM. Estimations indicate that the ratio between expelled and
retained comets ranges between $\eta = \frac{\text{Comets
expelled}}{\text{Comets retained}} = 3 \text{ - } 100$.

% TODO: **interstellar impostors** oorts cloud objects that resemble to interstellar objects
% TODO: review https://academic.oup.com/mnras/article/492/1/268/5621505?login=false
% https://www.sciencedirect.com/science/article/pii/S0019103523004232
