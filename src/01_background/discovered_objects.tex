\section{Discovered objects}

It was not until 2017 when the first interstellar object was discovered. The
object, named 1I/2017 U1, also known as 1I/'Oumuamua. Two years later, in 2019,
the second interstellar object was discovered. The object, named 2I/2019 Q4, and
referred to as 2I/Borisov. 

Despite having an interstellar origin, both objects presented different
properties. These properties are presented and discussed in the following
subsections.

\subsection{1I/'Oumuamua}

'Oumuamua was discovered on October 19, 2017, by the Pan-STARRS1 telescope in
Hawaii. It was first classified as a comet under the identifier of C/2017 U1,
but later reclassified as an an asteroid.

Its eccentricity was calculated to be around $1.20$, thus showing an hyperbolic
orbit. Its velocity was calculated to be close to $26.0$ km/s. Finally, it
entered the solar system with a direction $\alpha_{\text{ICRS}},\;
\delta_{\text{ICRS}} = 279^\circ.804,\; +33^\circ.997$, an inclination far from
the invariant plane of the solar system, see \cite{mamajek2017}. All these
attributes are consistent with an interstellar origin, as exposed in subsection
\ref{sec:expected_orbit_attributes}.


%\subsection{2I\/Borisov}
%
%...
%
%
%\subsection{Other candidates}
%
%...
