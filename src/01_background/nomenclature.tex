\section{Naming}

The naming of interstellar objects is imposed by the International Astronomical
Union (IAU). The IAU has established a nomenclature for interstellar objects,
which must follow the format\footnote{Exceptions exist. Identifiers may include
  the name of the discoverers or even popular names due to legacy reasons.}:

\begin{center}
  [Prefix]/[Year][Half-month][Number]
\end{center}

\textbf{Prefix} is a letter indicating the nature of the object according to
table \ref{tab:iau_prefixes}. Once the interstellar nature has been confirmed,
the prefix sticks to I.

\begin{table}[H]
  \centering
  \begin{tabular}{|c|c|}
    \hline
    \textbf{Object}     & \textbf{Prefix} \\
    \hline
    Comet               & C               \\
    Periodic comet      & P               \\
    Unknown orbit comet & X               \\
    Dissapeared comet   & D               \\
    Interstellar object & I               \\
    \hline
  \end{tabular}
  \caption{IAU prefixes for comets and interstellar objects.}
  \label{tab:iau_prefixes}
\end{table}

\textbf{Year} matches the number of the year of discovery while
\textbf{half-month} is a letter indicating the period of the year when the
discovery was made according to table \ref{tab:iau_half_month_id}.

\begin{table}[H]
  \centering
  \begin{tabular}{|c|c||c|c|}
    \hline
    \textbf{Latin Letter} & \textbf{Half-Month} & \textbf{Latin Letter} & \textbf{Half-Month} \\
    \hline
    A                     & Jan. 1-15           & B                     & Jan. 16-31          \\
    C                     & Feb. 1-15           & D                     & Feb. 16-29          \\
    E                     & Mar. 1-15           & F                     & Mar. 16-31          \\
    G                     & Apr. 1-15           & H                     & Apr. 16-30          \\
    J                     & May 1-15            & K                     & May 16-31           \\
    L                     & June 1-15           & M                     & June 16-30          \\
    N                     & July 1-15           & O                     & July 16-31          \\
    P                     & Aug. 1-15           & Q                     & Aug. 16-31          \\
    R                     & Sep. 1-15           & S                     & Sep. 16-30          \\
    T                     & Oct. 1-15           & U                     & Oct. 16-31          \\
    V                     & Nov. 1-15           & W                     & Nov. 16-30          \\
    X                     & Dec. 1-15           & Y                     & Dec. 16-31          \\
    \hline
  \end{tabular}
  \caption[IAU half-month identifier.]{IAU half-month identifier.}
  \label{tab:iau_half_month_id}
\end{table}

Finally, the \textbf{Number} is the digit representing the order of discovery
within the half-month of discovery. This number starts at $1$ for the first
discovered object of the half-month.
