\section{The role of the solar apex}

ISOs can present multiple directions when discovered. However, it is more likely
that they are discovered close to the direction of the solar apex, as it
happened with 'Oumuamua (constellation Lyra) and Borisov (constellation
Cassiopeia).

The solar apex, or the apex of the Sun's way, refers to the direction that the
Sun is moving with respect to the local standard of rest (LST). It is located in
the constellation Hercules, southwest of the bright star Vega and its visual
coordinates are right ascension 18h 28m 0s and declination +30° N. The solar
apex is moving at a speed of about 19.4 km/s (4.09 AU/year) relative to the
local standard of rest. On the other hand, the solar antapex, the direction
opposite the solar apex, is located near the star Zeta Canis Majoris in the
constellation Columba. 

Figure \ref{fig:solar_apex} shows the solar apex and antapex. The radial
velocities of stars in the solar neighborhood are represented by $V_r$ and their
proper motions are represented by $\mu$.

As the Sun moves towards the solar apex, nearby stars appear to be moving
away from that point in the sky. Conversely, stars appear to be moving closer
together in the direction of the solar antapex.

Note that the Sun's motion through the Milky Way galaxy is not confined to the
galactic plane, but also includes an oscillating motion relative to the plane
over millions of years.

\begin{figure}[H]
  \centering
  \includegraphics[width=0.5\textwidth]{static/solar_apex.png}
        \caption[The motion of the Sun in the LST.]
        {
          The motion of the Sun in the LST. The apex and antiapex are
          represented in the same and opposite direction of the Sun's motion.
          The combination of radial veolcities and proper motions of stars in the
          leads to the apparent motion of stars in the LST moving from the apex
          towards the antiapex.
        }
\label{fig:solar_apex}
\end{figure}
