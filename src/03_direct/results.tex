\section{Results}

Results obtained in tables \ref{tab:oumuamua-direct-transfer-impulses} and
\ref{tab:borisov-direct-transfer-impulses} show that the direct transfers
require small launch impulses, once the escape velocity of Earth is overcome.
However, they present large arrival velocities. This limits the amount of
observation time when performing a targeting with the ISOs.

Paying attention to figures \ref{fig:optimum_oumuamua_orbit_xz} and
\ref{fig:optimum_borisov_orbit_xz}, it can be identified that the transfer
orbits with the lowest launch energies reach the target when this is near the
ecliptic.

Bibliography proofs that the optimum direct transfer for borisov was already
computed by \cite{hibberd2021}. However, the optimum direct transfer for
'Oumuamua computed in this work could expand the research from \cite{hein2018}.

Notat that this analysis does not consider perturbation effects like:

\begin{itemize}
  \item Atmospheric drag at launch
  \item Solar radiation pressure
  \item Gravitational perturbations
  \item Spheres of influence (SOIs)
\end{itemize}

These effects could be considered in future works to improve the results
obtained for the direct transfers.
