\section{Characteristic energy at launch}

The characteristic energy for launch is the energy required to set a spacecraft
into the desired targeting orbit. This analysis assumes that the spacecraft
launches from Earth, which is modeled as a point with no mass and arrives at the
target interloper, modeled again as a massless point. The only force acting on
the spacecraft is the gravitational force of the Sun.

Before launching, the spacecraft has the velocity of Earth, $\vec{v_{\oplus}}$.
At launch, the spacecraft presents an heliocentric velocity $\vec{v_{\infty
        ,1}}$ that matches the solution of Lambert's problem. Vector $\Delta{\vec{v}}$
is the difference between the two velocities, and its modulus matches the
required impulse velocity, as stated in Equation \ref{eq:launch_velocity}.

\begin{equation}
  \Delta v_1 = \|\vec{v_{\infty ,1}} - \vec{v_{\oplus}}\|
  \label{eq:launch_velocity}
\end{equation}

The value of $\Delta v_1$ can be used in Equation \ref{eq:c3} for solving the
characteristic for launch.

\subsection{1I/'Oumuamua}

Porkchop plots for 1I/'Oumuamua representing the characteristic energy for launch
are shown in figure \ref{fig:oumuamua-direct-prograde-transfer-porkchop} and
figure \ref{fig:oumuamua-direct-retrograde-transfer-porkchop}.

\begin{figure}[H]
  \centering
  \includegraphics[width=\textwidth]{static/oumuamua/direct-prograde-transfer-porkchop.png}
  \caption[Direct and prograde launch energy porkchop for 1I/'Oumuamua]{Launch energy porkchop plot for 1I/'Oumuamua for a direct and prograde
    transfer showing the isolines for
    the time of flight required for a targeting. A region of low transfer
    energy is located in the lower left corner.}
  \label{fig:oumuamua-direct-prograde-transfer-porkchop}
\end{figure}

\begin{figure}[H]
  \centering
  \includegraphics[width=\textwidth]{static/oumuamua/direct-retrograde-transfer-porkchop.png}
  \caption[Direct and retrograde launch energy porkchop for
    1I/'Oumuamua]{Launch energy porkchop plot for 1I/'Oumuamua for a direct and
    retrograde transfer showing the isolines for
    the time of flight required for a targeting.}
  \label{fig:oumuamua-direct-retrograde-transfer-porkchop}
\end{figure}

In figures \ref{fig:oumuamua-direct-prograde-transfer-porkchop} and
\ref{fig:oumuamua-direct-retrograde-transfer-porkchop}, shorter time of flights
require higher characteristic energies for launch. One may think that a
retrograde transfer orbit would be more efficient considering that 1I/'Oumuamua has
this kind of inclination. However, retrograde orbits do not benefit from Earth's
velocity at launch, which makes them less efficient. Therefore, characteristic
energies for retrograde transfers are higher than for prograde transfers.

Both porkchop plots present a pattern whose solutions alternate between low and
high characteristic energies. This pattern is a consequence of the relative
position of the earth with respect to the target. Low energy solutions
correspong to positions in which the velocity of the Earth gets aligned with the
launch velocity vector.

Note that, despite some areas in the figures not being colored, they have a
solution. These areas do not show any color due to the upper limit for the characteristic
energy at launch imposed of $C_3 = 10000$ km$^2$/s$^2$, imposed by the
author. This allows for a better representation of the porkchops to visually
identify regions. This allows to identify a small region, between years 2016 and
2018 where values for $C_3$ are lower.

\subsection{2I/Borisov}

Porkchop plots for 2I/Borisov representing the characteristic energy for launch
are shown in figure \ref{fig:borisov-direct-prograde-transfer-porkchop} and
figure \ref{fig:borisov-direct-retrograde-transfer-porkchop}. These figures
remember to the ones for 1I/'Oumuamua, although they present different numerical
values. Again, the shorter the time of flight, the greater the characteristic
energy for launching the spacecraft. Retrograde transfers, once again, are less
efficient than prograde and a pattern of low and high energy solutions is
present.

Values for 2I/Borisov's characteristic energy at launch are lower than 1I/'Oumuamua
for the same time of flight. Despite having a greater relative velocity, 2I/Borisov
has a lower inclination than 1I/'Oumuamua with respect to the ecliptic. Thanks to
this low inclination, a spacecraft departing from Earth can benefit a bit more
from Earth's velocity at launch time, reducing the required energy.

As a result of previous situation, a set of low energy transfers appears between
years 2016 and 2020. However, as opposite to 1I/'Oumuamua, these seem to extend a
bit further in the direction of the arrival date. This region is analyzed in
detail in the next section.

\begin{figure}[H]
  \centering
  \includegraphics[width=\textwidth]{static/borisov/direct-prograde-transfer-porkchop.png}
  \caption[Direct and prograde launch energy porkchop for
    2I/Borisov]{Launch energy porkchop plot for 2I/Borisov for a direct and prograde
    transfer showing the isolines for
    the time of flight required for a targeting. Values under 1000
    km$^2$/s$^2$ show in the lower left corner. This region should be
    further explored.}
  \label{fig:borisov-direct-prograde-transfer-porkchop}
\end{figure}

\begin{figure}[H]
  \centering
  \includegraphics[width=\textwidth]{static/borisov/direct-retrograde-transfer-porkchop.png}
  \caption[Direct and retrograde launch energy porkchop for
    2I/Borisov]{Launch energy porkchop plot for 2I/Borisov for a direct and
    retrograde transfer showing the isolines for
    the time of flight required for a targeting. The retrograde case shows
    energies too large for a suitable transfer.}
  \label{fig:borisov-direct-retrograde-transfer-porkchop}
\end{figure}

