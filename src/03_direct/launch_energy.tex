\section{Characteristic energy for launch}

The characteristic energy for launch is the energy required to set a spacecraft
into the desired targeting orbit. This analysis assumes that the spacecraft
launches from Earth, which is modeled as a point with no mass and arrives at the
target interloper, modeled again as a massless point. The only force acting on
the spacecraft is the gravitational force of the Sun.

Before launching, the spacecraft has the velocity of Earth, $\vec{v_{\oplus}}$.
At launch, the spacecraft presents an heliocentric velocity $\vec{v_{\infty
,1}}$ that matches the solution of Lambert's problem. Vector $\Delta{\vec{v}}$
is the difference between the two velocities, and its modulus matches the
required impulse velocity, as stated in Equation \ref{eq:launch_velocity}.

\begin{equation}
    \Delta v_1 = \|\vec{v_{\infty ,1}} - \vec{v_{\oplus}}\|
    \label{eq:launch_velocity}
\end{equation}

The value of $\Delta v_1$ can be used in Equation \ref{eq:c3} for solving the
characteristic for launch.

\subsection{'Oumuamua}

Porkchop plots for 1I/'Oumuamua representing the characteristic energy for launch
are shown in figure \ref{fig:oumuamua-direct-prograde-transfer-porkchop} and
figure \ref{fig:oumuamua-direct-retrograde-transfer-porkchop}.

\begin{figure}[H]
  \centering
  \includegraphics[width=\textwidth]{static/oumuamua/direct-prograde-transfer-porkchop.png}
        \caption[Direct and prograde launch energy porkchop for 'Oumuamua]{Launch energy porkchop plot for 1I/'Oumuamua for a direct and prograde
        transfer showing the isolines for
        the time of flight required for a targeting.}
  \label{fig:oumuamua-direct-prograde-transfer-porkchop}
\end{figure}

\begin{figure}[H]
  \centering
  \includegraphics[width=\textwidth]{static/oumuamua/direct-retrograde-transfer-porkchop.png}
        \caption[Direct and retrograde launch energy porkchop for
        'Oumuamua]{Launch energy porkchop plot for 1I/'Oumuamua for a direct and
        retrograde transfer showing the isolines for
        the time of flight required for a targeting.}
  \label{fig:oumuamua-direct-retrograde-transfer-porkchop}
\end{figure}

In figures \ref{fig:oumuamua-direct-prograde-transfer-porkchop} and
\ref{fig:oumuamua-direct-retrograde-transfer-porkchop}, shorter time of flights
require higher characteristic energies for launch. One may think that a
retrograde transfer orbit would be more efficient considering that 'Oumuamua has
this kind of inclination. However, retrograde orbits do not benefit from Earth's
velocity at launch, which makes them less efficient. Therefore, characteristic
energies for retrograde transfers are higher than for prograde transfers.

Both porkchop plots present a pattern whose solutions alternate between low and
high characteristic energies. This pattern is a consequence of the relative
position of the earth with respect to the target. Low energy solutions
correspong to positions in which the velocity of the Earth gets aligned with the
launch velocity vector.

\subsection{Borisov}

Porkchop plots for 2I/Borisov representing the characteristic energy for launch
are shown in figure \ref{fig:borisov-direct-prograde-transfer-porkchop} and
figure \ref{fig:borisov-direct-retrograde-transfer-porkchop}.

\begin{figure}[H]
  \centering
  \includegraphics[width=\textwidth]{static/borisov/direct-prograde-transfer-porkchop.png}
        \caption[Direct and prograde launch energy porkchop for
        2I/Borisov]{Launch energy porkchop plot for 2I/Borisov for a direct and prograde
        transfer showing the isolines for
        the time of flight required for a targeting.}
  \label{fig:borisov-direct-prograde-transfer-porkchop}
\end{figure}

\begin{figure}[H]
  \centering
  \includegraphics[width=\textwidth]{static/borisov/direct-retrograde-transfer-porkchop.png}
        \caption[Direct and retrograde launch energy porkchop for
        2I/Borisov]{Launch energy porkchop plot for 2I/Borisov for a direct and
        retrograde transfer showing the isolines for
        the time of flight required for a targeting.}
  \label{fig:borisov-direct-retrograde-transfer-porkchop}
\end{figure}

These figures remember to the ones for 1I/'Oumuamua although present different
numerical values. Again, the shorter the time of flight, the greater the
characteristic energy for launch. Also, retrograde transfers are less efficient
than prograde and a pattern of low and high energy solutions is present.

The main difference between the porkchop plots for 'Oumuamua and Borisov are the
values. Overall, these are higher for Borisov than for 'Oumuamua. This is a
consequence of the high velocity of Borisov with respect to the Sun. Due to this
situation, a spacecraft targeting Borisov requires more energy to match its
speed.

However, a closer look at the prograde porkchop plots when limiting the characteristic
launch energy up to $C_3 = 625.00$ km$^2$/s$^2$, equivalent to $\Delta v =
25.00$ km/s, reveals an interesting pattern as presented in figure
\ref{fig:borisov-direct-detailed-porkchop}.

\begin{figure}[H]
  \centering
  \includegraphics[width=\textwidth]{static/borisov/direct-detailed-porkchop.png}
        \caption[Detailed porkchop showing the optimum transfer for
        2I/Borisov]{The porkchop reveals a series of optimum transfer orbits.
        These values remain suitable for modern chemical propulsion media.}
  \label{fig:borisov-direct-detailed-porkchop}
\end{figure}
