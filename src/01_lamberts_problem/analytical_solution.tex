\section{The analytical solution to the problem}

Because reader was introduced previously to the series solution proposed by
Lambert in \ref{eq:visviva_series} together with the geometrical background of
the problem, an analytical solutions has been collected to give a complete
understanding of the problem.

Although Lambert is considered the be first one to address the BVP and made the
first contributions to its solution, the first to formally solve the Lambert's
problem was Joseph-Louis Lagrange. Other authors such us Carl Friedrich Gauss
also provided analytical solutions, which he used for the computation of the
orbit of Ceres.

Therefore in the following lines, the solutions devised by Lagrange and Gauss
are included with the purpose of giving a complete vision of the Lambert's
problem from the classical analysis.

\subsection{Lagrange's solution}

Lagrange's procedure starts with Kepler's equation for the elliptic
case\footnote{Notice that this is not the ideal procedure as the type of orbit
  is not known. A particular form of the Kepler's equation can be applied
  only if the observer is sure about the type of motion of the orbiting
  body.}

\begin{equation}
  \sqrt{\frac{\mu}{a^3}} \Delta t = E_{2} - E_{1} - e \cdot \left(\sin{(E_{2}) -\sin{E_{1}}}) \right)
  \label{eq:kepler_lagrange}
\end{equation}

where in previous equation $E$ refers to the eccentric anomaly and the
subscripts $1$ or $2$ the evaluation point (initial or final position vector).In
addition, applying the radial distance equation, it is possible to find:

\begin{align}
  \norm{\vec{r_{1}}} & = a \cdot (1 - e \cdot \cos{(E_1)}) \\
  \norm{\vec{r_{2}}} & = a \cdot (1 - e \cdot \cos{(E_2)}) \\
  \label{eq:radial_distances_lagrage}
\end{align}

If relating the true anomaly $\nu$ and the eccentric one $E$ for each one of the
points, another two relations arise:

\begin{align}
  \cos{(\nu_{1} - w)} & = \frac{a}{\norm{\vec{r_{1}}}} \cdot \left(\cos{(E_1) - e} \right) \\
  \cos{(\nu_{2} - w)} & = \frac{a}{\norm{\vec{r_{2}}}} \cdot \left(\cos{(E_2) - e} \right)
  \label{eq:ecc_to_true_lagrange}
\end{align}

being $w$ the reference for true anomaly measurements, that is the argument of
periapsis.

The set of five expressions given by equations \ref{eq:kepler_lagrange},
\ref{eq:radial_distances_lagrage} and $\ref{eq:ecc_to_true_lagrange}$ contains a
total of five unknowns: $a$, $e$, $w$, $E_{1}$ and $E_{2}$. However, only
the first three ones are required for the determination of the shape of the
orbit.

To simplify these expressions, Lagrange introduced two new expressions:

\begin{equation}
  \psi = \frac{E_2 - E_1}{2}\quad\quad
  \cos{(\varphi)} = e \cdot \cos{\left(\frac{E_2 + E_1}{2} \right)}
\end{equation}

and related them via two new angles:

\begin{equation}
  \alpha = \varphi + \psi\quad\quad
  \beta = \varphi - \psi
\end{equation}

With all these relations, equation \ref{eq:kepler_lagrange} becomes:

\begin{equation}
  \sqrt{\frac{\mu}{a^3}} \Delta t = \alpha - \beta - \left(\sin{(\alpha)} - \sin{(\beta)}\right)
  \label{eq:kepler_lagrange_simple}
\end{equation}

where it can be proof that $\alpha$ and $\beta$ are direct functions of the
semi-major axis of the orbit such that:

\begin{equation}
  \sin{\left(\frac{\alpha}{2} \right)} = \sqrt{\frac{s}{2a}\quad\quad}
  \sin{\left(\frac{\beta}{2} \right)} = \sqrt{\frac{s - c}{2a}\quad\quad}
  \label{eq:alpha_beta}
\end{equation}

being $s$ and $c$ the semi-perimeter and chord respectively, as in equation
\ref{eq:s_and_c_lambert}.

Because of the ambiguity of the solution, the values of $\alpha$ and $\beta$
need to be fixed depending on some conditions of the problem such that. For
$\alpha$, the correction to be applied is:

\begin{equation}
  \alpha = 2 \arcsin{\sqrt{\frac{s}{2a}}}
\end{equation}

as long as $\alpha < 2\pi$. If previous condition does not hold, the angle is
obtained such that $\alpha = 2\pi - \alpha$.

For the case of $\beta$, its value is computed regarding the one for the
transfer angle $\Delta \theta$:

\begin{equation}
  \beta = 2 \arcsin{\sqrt{\frac{s - c}{2a}}}
\end{equation}

as long as $\Delta \theta < \pi$. If not, then a sign correction is applied such
that $\beta = -\beta$.

Equation \ref{eq:kepler_lagrange_simple} can be obtained by using a root solver
so the right value for $a$ is computed. Once the semi-major axis has been
solved, it is possible to compute other orbit elements by using:

\begin{equation}
  e = \sqrt{\left( \frac{\norm{\vec{r_{1}}} +
      \norm{\vec{r_{2}}}}{2a\sin{\psi}} \right)^{2} + \left(\cos{\varphi} \right)^2}
\end{equation}

\subsection{Gauss' solution}

The method developed by Gauss to the determination of the Orbit of Ceres was
published in his \citetitle{gauss1809}. It must be pointed out that the method
presented in the following lines is singular for transfer angles of $180$
degrees and the rate of converge is slow when its value is not small. However,
the method became very popular in its days and is set up the basis for new ones.

The relations presented in the next lines follow the notation employed by
\citeauthor{bate1979} in their book \citetitle{bate1979}.

Gauss' method is strongly related with the so-called \textit{ratio of sector to
  triangle}. This is nothing but Kepler's second law\footnote{Kepler's second law
  states that an orbiting body sweeps out equal areas in equal lengths of time.}
The relation modeling this fact is as follows:

\begin{equation}
  dt = \frac{2}{h}dA
\end{equation}

which evaluated at $h=\sqrt{\mu p}$ (being $p$ the orbital parameter) gives the
area of the sector:

\begin{equation}
  A_{s} = \frac{1}{2} \sqrt{\mu p} \Delta t
  \label{eq:gauss_area_sector}
\end{equation}

On the other hand, the area of the triangle $A_{t}$ formed by the two position
vectors and the chord can be obtained via geometrical relations, adopting the
form:

\begin{equation}
  A_{t} = \frac{1}{2} \norm{\vec{r_{1}}}  \norm{\vec{r_{2}}} \sin{(\Delta \theta)}
  \label{eq:gauss_area_triangle}
\end{equation}

Gauss related equations \ref{eq:gauss_area_sector} and
\ref{eq:gauss_area_triangle}, so that:

\begin{equation}
  y = \frac{A_s}{A_t} = \frac{\sqrt{\mu p} \Delta t}{\norm{\vec{r_{1}}}  \norm{\vec{r_{2}}} \sin{(\Delta \theta)}}
  \label{eq:gauss_y}
\end{equation}

Notice that the only unknown in previous equation is the orbit parameter $p$.

The goal now is to relate previous variable $y$ with the value of $\Delta E$.
Hopefully, the following expression solves for this issue:

\begin{equation}
  p = \frac{\norm{\vec{r_1}} \norm{\vec{r_2}} (1 - \cos{\Delta
      \theta})}{\norm{\vec{r_1}} + \norm{\vec{r_2}} - 2
    \sqrt{\norm{\vec{r_1}} + \norm{\vec{r_2}} \cos{\left(\frac{\Delta
          \theta}{2}\right)} \cos{\left(\frac{\Delta E}{2} \right)}}}
  \label{eq:p_as_of_E}
\end{equation}

Two new relations are introduced:

\begin{equation}
  s = \frac{\norm{\vec{r_1}} + \norm{\vec{r_2}}}{4\sqrt{\norm{\vec{r_1}}
      \norm{\vec{r_2}}}\cos{\left(\frac{\Delta \nu}{2} \right)} } - \frac{1}{2}
  \label{eq:gauss_s}
\end{equation}

and:

\begin{equation}
  w = \frac{\mu \Delta t^2}{\left(2\sqrt{\norm{\vec{r_1}}\norm{\vec{r_2}}} \cos\left(\frac{\Delta \nu}{2} \right )\right)^3}
  \label{eq:gauss_w}
\end{equation}

Be careful not to confuse equation \ref{eq:gauss_s} with the relation for the
semi-perimeter. Finally, by replacing the relations \ref{eq:p_as_of_E},
\ref{eq:gauss_s} and \ref{eq:gauss_w} into \ref{eq:gauss_y} it is possible to
obtain:

\begin{equation}
  y^2 = \frac{w}{s + \frac{1}{2}\left(1 - \cos{\left(\frac{\Delta E}{2} \right)} \right)}
  \label{eq:gauss_first_eq}
\end{equation}

Equation \ref{eq:gauss_first_eq} is known as \textit{Gauss' first equation}. Two
unknowns appear in this equation, being $y$ and $\Delta E$. Therefore, another
equation is required in order to determine these two values.

The process for getting this second equation is more complex and involves the
$f$ and $g$ formulation. The whole procedure is properly explained in
\cite{bate1979}, so only the theoretical background will be pointed out here
together with the final expression.

Gauss was able to combine in a very smart way equation \ref{eq:gauss_y} with the
$f$ and $g$, so an expression depending only on $y$ and $\Delta E$ appeared:

\begin{equation}
  y = 1 + \left(\frac{\Delta E - \sin{(\Delta E)}}{\left(\sin{\left(\frac{\Delta E}{2} \right )} \right)^3} \right )\left(s + \frac{1 - \cos{\left(\frac{\Delta E}{2} \right )}}{2} \right )
  \label{eq:gauss_second_eq}
\end{equation}

where \ref{eq:gauss_second_eq} is known as \textit{Gauss' second equation}.
Because this equation was obtained using the \textit{ratio of sector to triangle
  area} (equation \ref{eq:gauss_y}) the method developed by Gauss is also named
like this.

Notice that the two equations \ref{eq:gauss_first_eq} and
\ref{eq:gauss_second_eq} only have as unknowns $y$ and $\Delta E$. Once this
system of equations is known by applying a numerical method to a particular
tolerance, the rest of the orbit parameter can be obtained using equation
\ref{eq:p_as_of_E}. In addition, $f$ and $g$ functions can be solved so that the
velocity vectors are found to be:

\begin{equation}
  \vec{v_1} = \frac{\vec{r_2} - f \vec{r_1}}{g}\quad\quad
  \vec{v_2} = \dot{f} \vec{r_1} + \dot{g} \vec{v_1}
\end{equation}


