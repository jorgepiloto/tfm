\section{The analytical solution to the problem}

Although Lambert is considered the be first one to formulate the BVP of the
two-body two-point problem he only provided the relation between the time of
flight and the geometry of the orbit. No procedure or solution is known to be
published by him, even if he ever claimed this to Euler.


\subsection{Lagrange's solution}
\label{sec:lagrange_sol}

Some years, after the publication of Lambert's book, Joseph Louis Lagrange
published \citetitle{lagrange1788} in which the first procedure to solve
Lambert's problem appeared for the first time.

In the following lines, his method is presented so the reader has a modern
vision on this classical solver. The procedure presented in here is the one
introduced by \cite{jiang2016}, who carefully revised the algorithm. Only the
fundamental equations of the problem will be collected here, so reader is
encouraged to review Jiang's article for a deep understanding.

Recalling Kepler's equation and Lambert's theorem, it is possible to obtain
equation \ref{eq:lagrange_tof} between the time of flight and the geometry of
the problem:

\begin{equation}
  \sqrt{\mu} \Delta t = F(\norm{\vec{r_1}} + \norm{\vec{r_2}}, \norm{\vec{c}}, a)
  \label{eq:lagrange_tof}
\end{equation}

where the function $F$ depends on the geometry of the orbit, which is not known
at first sight. To do so, the time of flight for a pure parabolic transfer is
computed using the first term of the series presented in \ref{eq:visviva_series}
or in a more compact way:

\begin{equation}
	\Delta t_p = \frac{1}{\sqrt{\mu}} \left(
	\frac{1}{6}\left(\norm{\vec{r_1}} + \norm{\vec{r_2}} + \norm{\vec{c}}
\right)  \pm \left(\norm{\vec{r_1}} + \norm{\vec{r_2}} - \norm{\vec{c}} \right) \right)
\end{equation}

where the $\pm$ indicates if the transfer angle is lower or greater than $\pi$
respectively.

Knowing the parabolic time of flight, it is possible to know if the orbit being
addressed is elliptic or hyperbolic by direct comparison with the current time
of flight:

\begin{itemize}
  \item If $\Delta t > \Delta t_p$, then the orbit is \textit{elliptic}.
  \item If $\Delta t < \Delta t_p$, then the orbit is \textit{hyperbolic}.
\end{itemize}

Once the nature of the transfer orbit is known, the value of $F$ can be
replaced by the corresponding Kepler's equation:

\begin{equation}
  F = 
  \begin{cases}
  \sqrt{\mu} \Delta t_e = a^{\frac{3}{2}}\left((\alpha - \sin{(\alpha)}) + (\beta - \sin{(\beta)}) \right) \\
  \sqrt{\mu} \Delta t_h = (-a)^{\frac{3}{2}}\left((\sinh{(\alpha)} - \alpha) + (\sinh{(\beta)} - \beta) \right) \\
  \end{cases}
  \label{eq:tof_F_lagrange}
\end{equation}

where in equation \ref{eq:tof_F_lagrange}, the angles $\alpha$ and $\beta$ were
introduced by Lagrange. To compute these angles, first the auxiliary values
$\alpha_0$ and $\beta_0$ are found using:

\begin{equation}
  \alpha_0 = 2 \cdot \arccos{\sqrt{\frac{s}{2a}}}\quad\quad\quad
  \beta_0 = 2 \cdot \arccos{\sqrt{\frac{s - \norm{\vec{c}}}{2a}}}
\end{equation}

for the elliptic case and:

\begin{equation}
  \alpha_0 = 2 \cdot \arccosh{\sqrt{\frac{s}{-2a}}}\quad\quad\quad
  \beta_0 = 2 \cdot \arccosh{\sqrt{\frac{s - \norm{\vec{c}}}{-2a}}}
\end{equation}

for the hyperbolic one.

Once computed, a correction needs to be performed depending on the transfer
angle:

\begin{equation}
  \alpha =
  \begin{cases}
	  0 \leqslant \alpha_0 \leqslant 2\pi & \text{for elliptic/hyperbolic orbits always} \\
  \end{cases}
\end{equation}

and:

\begin{equation}
  \beta =
  \begin{cases}
	  0 \leqslant \beta_0 \leqslant \pi & \text{for elliptic orbits if $\Delta \theta \leqslant \pi$} \\
	  -\pi \leqslant \beta_0 \leqslant 0 & \text{for elliptic orbits if $\Delta \theta > \pi$} \\
	  0 \leqslant \beta_0 & \text{for hyperbolic orbits if $\Delta \theta \leqslant \pi$} \\
	  \beta_0 \leqslant 0 & \text{for hyperbolic orbits if $\Delta \theta > \pi$}
  \end{cases}
\end{equation}

If reader is interested in the physical meaning of this angles, a deep study was
made by \cite{prussing1979}.

Finally, an initial guess about $a$ is made, named $a_0$. By substituting in the
proper equation for the transfer geometry, the function for $F$ is compared with
the current value of Kepler's equation for the real time of flight such that:

\begin{equation}
	\sqrt{\mu} \Delta t - F = 0
	\label{eq:sol_lagrange}
\end{equation}

Equation \ref{eq:sol_lagrange} might not be equal to zero but close to this
value. Therefore, once it below a desired accuracy, the proper value of $a$ has
been solved and the values $p$, the semi-latus rectum, can be computed using:

\begin{equation}
 p = 
  \begin{cases}
  \frac{4a(s - \norm{\vec{r_1}})(s - \norm{\vec{r_2}})}{\norm{\vec{c}}^2} \left(\arcsin{\left(\frac{\alpha + \beta}{2} \right)}\right)^{2} & \text{for elliptic orbits}\\ 
  \frac{-4a(s - \norm{\vec{r_1}})(s - \norm{\vec{r_2}})}{\norm{\vec{c}}^2} \left(\arcsinh{\left(\frac{\alpha + \beta}{2} \right)}\right)^{2} & \text{for hyperbolic  orbits}\\
  \end{cases}
\end{equation}

with this new orbit parameter, the rest can be computed. Finally, a simple
transformation from COE to RV can be applied in order to get the initial and
final position vectors although \cite{jiang2016} provides direct expressions
based on the semi-latus rectum.
