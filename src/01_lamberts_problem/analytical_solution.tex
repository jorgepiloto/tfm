\section{The analytical solution to the problem}

Because reader was introduced previously to the series solution proposed by
Lambert in \ref{eq:visviva_series} together with the geometrical background of
the problem, an analytical solutions has been collected to give a complete
understanding of the problem.

Although Lambert is considered the be first one to address the BVP and made the
first contributions to its solution, the first to formally solve the Lambert's
problem was Joseph-Louis Lagrange. Other authors such us Carl Friedrich Gauss
also provided analytical solutions to the problem although they have not been
included in this work as abundant literature on them are available.

Therefore in the following lines, the approach devised by Lagrange is included
only with the purpose of giving a complete vision of the Lambert's problem from
the classical analysis.

\subsection{Lagrange's solution}

Lagrange's procedure starts with Kepler's equation for the elliptic
case\footnote{Notice that this is not the ideal procedure as the type of orbit
  is not known. A particular form of the Kepler's equation can be applied
  only if the observer is sure about the type of motion of the orbiting
  body.}

\begin{equation}
  \sqrt{\frac{\mu}{a^3}} \Delta t = E_{2} - E_{1} - e \cdot \left(\sin{(E_{2}) -\sin{E_{1}}}) \right)
  \label{eq:kepler_lagrange}
\end{equation}

Other useful relations are:

\begin{align}
  \norm{\vec{r_{1}}}        & = a \cdot (1 - e \cdot \cos{(E_1)})                                \\
  \norm{\vec{r_{2}}}        & = a \cdot (1 - e \cdot \cos{(E_2)})                                \\
  \cos{(\nu_{1} - \nu_{0})} & = \frac{a}{\norm{\vec{r_{1}}}} \cdot \left(\cos{(E_1) - e} \right) \\
  \cos{(\nu_{2} - \nu_{0})} & = \frac{a}{\norm{\vec{r_{2}}}} \cdot \left(\cos{(E_2) - e} \right)
\end{align}

where in previous equations, $E_{i}$ are the eccentric anomalies and $\nu_{i}$
the true anomalies. The parameter $\nu_{0}$ is the reference one for measuring
the true anomalies, that is, the semi-major axis line.

Lagrange introduced the variables:

\begin{equation}
  \psi = \frac{E_2 - E_1}{2}\quad\quad
  \cos{(\varphi)} = e \cdot \cos{\left(\frac{E_2 + E_1}{2} \right)}
\end{equation}

and related them via two new angles:

\begin{equation}
  \alpha = \varphi + \psi\quad\quad
  \beta = \varphi - \psi
\end{equation}

With all these relations, equation \ref{eq:kepler_lagrange} becomes:

\begin{equation}
  \sqrt{\frac{\mu}{a^3}} \Delta t = \alpha - \beta - \left(\sin{(\alpha)} - \sin{(\beta)}\right)
  \label{eq:kepler_lagrange_simple}
\end{equation}

where it can be proof that $\alpha$ and $\beta$ are direct functions of the
semi-major axis of the orbit such that:

\begin{equation}
  \sin{\left(\frac{\alpha}{2} \right)} = \frac{s}{2a}\quad\quad
  \sin{\left(\frac{\beta}{2} \right)} = \frac{s - c}{2a}\quad\quad
  \label{eq:alpha_beta}
\end{equation}

By combining equations \ref{eq:kepler_lagrange_simple} and \ref{eq:alpha_beta}
it is possible to obtain a single expression dependent only on $a$ and known
parameters of the problem. This new equation can be solved by making use of a
root solver to find the current value of the semi-major axis. Once it is known,
the rest of the orbital elements can be properly computed.
