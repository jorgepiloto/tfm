\section{Mission constraints}

The characteristic energy $C_3$ is a good estimator for the propulsion
requirements of a mission. If a propulsion system is imposed, its maximum
specific energy can be retrieved and imposed as a mission constraint to know
which targeting orbits can be followed.

Another mission constraint within the context of interlopers rendezvous is the
excess velocity at arrival. The faster the rendezvous, the greater the thrust
required for slowing down the spacecraft. If the final impulse is not applied to
adapt the spacecraft to the orbit of the interloper, then the excess velocity at
arrival limits the amount of time available for observation and data collection.

Finally, other constraints such as the time of flight, the launch window,
tracking or communication constraints can be imposed to further refine the
analysis.

Some of these mission constraints can be visualized in a contour map relating
the characteristic energy required for a given combination of launch and arrival
dates. These maps are called porkchop plots and are useful to identify the most
optimal transfer windows for a given mission.
