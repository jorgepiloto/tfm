\section{Mission constraints}

Despite simplifications, the mission design process is still a complex problem.
There exists a wide range of mission constraints that can be imposed on the analysis
to refine the results and make them more realistic. These include:

\begin{itemize}
    \item Fuel mass
    \item Characteristic energy
    \item Excess velocity at arrival
    \item Time of flight
    \item Tracking constraints
    \item Communication constraints
\end{itemize}

\subsection{Fuel mass}

The fuel mass is a critical constraint in mission design. For every impulse
performed by the spacecraft, a certain amount of fuel is consumed. This loss in
mass is modeled according to the Tsiolkovsky rocket equation:

\begin{equation}
        \Delta v = v_e \ln \left( \frac{m_0}{m_f} \right) = I_{sp} g_0 \ln \left( \frac{m_0}{m_f} \right)
        \label{eq:tsiolkovsky}
\end{equation}

Where $\Delta v$ is the change in velocity, $v_e$ is the exhaust velocity, $m_0$
is the initial mass of the spacecraft, $m_f$ is the final mass of the
spacecraft. Other variants of the expression include the $I_{sp}$, the specific
impulse of the propulsion system, and $g_0$, the standard gravity.

\subsection{Characteristic energy}

The characteristic energy $C_3$ is also a good estimator for the propulsion
requirements of a mission.

The $\Delta v$ relates with the characteristic energy $C_3$, also known as
specific energy. Equation \ref{eq:c3} summarizes this relation for hyperbolic
orbits:

\begin{equation}
        C_3 = v_{\infty}^2
        \label{eq:c3}
\end{equation}

Given a propulsion system, a maximum specific energy is imposed, limiting the
maximum $\Delta v$ that the spacecraft can achieve. If a spacraft can not reach
a certain characteristic energy, then the mission is not feasible and the
target orbit can not be achieved.

\subsection{Excess velocity at arrival}

Another mission constraint within the context of interlopers rendezvous is the
excess velocity at arrival. Lauch and arrival velocities are used to compute the
impulsed required to reach the target orbit. The first impulse $\Delta v_1$ is
used to launch the spacecraft into the target orbit. The second impulse $\Delta
v_2$ is used to adapt to the orbit of the interloper, leading to a rendezvous.
Thus, two scenarios are possible:

\begin{itemize}

    \item \textbf{Targeting of the interloper.} The spacecraft overshoots the target by
    not applying the final impulse. This ahieves a greater launch
    impulse as more fuel mass can be allocated for this task. However, the
    spacecraft is not able to rendezvous with the interloper, as the second
    impulse is never applied.

    \item \textbf{Rendezvous of the interloper.} The spacecraft performs the
    arrival impulse to adapt to the orbit of the interloper. This reduces the
    amount of fuel available for the launch impulse but allows the spacecraft to
    follow the interloper.

\end{itemize}

Wether the spacecraft overshoots the target or rendezvous with the interloper,
the excess velocity at arrival is a critical parameter. 
