\section{Problem description and motivation}

The discovery of interstellar objects such as 'Oumuamua and Borisov within our
solar system has ignited a surge of curiosity and scientific interest. These
sub-kilometer-sized visitors, originating from distant stellar systems, present
a unique opportunity to study extraterrestrial bodies that have traversed vast
cosmic distances. Therefore, the main motivations behind the study of these
objects are:

\begin{itemize}

  \item \textbf{Better understanding the formation of planetary systems.}
        Interstellar objects can provide insights into the formation and
        dynamics of planetary systems beyond our own. This could help to
        confirm or reject the Nebular hyphotesis, which is the most popular
        model proposed for the formation of planetary systems by in situ
        measurements of the isotopic signatures.

  \item \textbf{Exploring the origins of life.} Analyzing the composition of
        interstellar objects could provide valuable information about the
        chemical and physical conditions present in other planetary
        systems, shedding light on the origins of life in the universe, which
        could support or reject the panspermia hyphotesis.

  \item \textbf{Technological innovation.} By pushing the technological
        boundaries of space exploration, missions to intercept interstellar
        objects could lead to the development of new propulsion systems and
        spacecraft capable of reaching unprecedented speeds and distances.

\end{itemize}

Given their exceptionally high eccentricities, heliocentric velocities, and
fleeting passage through the planetary region, there is a pressing need for the
development of ready-to-launch missions capable of intercepting them.

However, the design of such missions is not straightforward. The high velocities
of these objects, combined with their limited observation windows, make it
difficult to accurately predict their trajectories and plan for rendezvous
within the short timeframes available.

This problem presents the main motivation of this work: \textbf{devising mission
  orbits capable of intercepting interstellar objects}.

This research stems from the desire to unlock the mysteries surrounding these
enigmatic interstellar travelers, gathering invaluable data or even returning
samples from their surfaces. This pursuit not only promises to broaden our
understanding of celestial dynamics and planetary formation but also holds
profound implications for the future of space exploration and our comprehension
of the broader universe.
