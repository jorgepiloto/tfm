\section{Objectives and goals}

The main objetive of this project is to design suitable targeting orbits for
interstellar ojects. To achieve this purpose, the whole process is divided
into the following key objectives:

\begin{itemize}

  \item \textbf{Research on interstellar objects.}
        The definition of interstellar object is presented together with the
        official IAU nomenclature. The only two discovered objects,
        1I/'Oumuamua and 2I/Borisov, are presented together with their main
        characteristics. The importance of the solar apex is explained
        too.

  \item \textbf{Revisiting targeting orbits.}
        Lambert's problems is revisited. Porkchop plots are presented to the
        reader together with other useful mission design tools. These are
        applied to the two discovered interstellar objects, demonstrating the
        targeting challenges considering nowadays technologies.

  \item \textbf{Alternate targeting orbits}
        Once the direct transfer problems have been introduced, the concept of
        Lagrangian points is presented. The analysis focuses on point L2, due
        to its capabilities and advantages. The gravity assist technique is
        revisited and explored.

  \item \textbf{Performance comparison.}
        A comparison between direct transfers and multiple flyby orbits is
        performed. Mission contrains including velocity impulse, time of
        flight, arrival excess velocity, and existing propulsive technologies
        are considered.

  \item \textbf{Results}
        Advantages and disadvantages of each mission are presented. Improvements
        to the custom flyby algorithm are also discussed.
\end{itemize}
