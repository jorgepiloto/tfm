\section{Classification and inheritance diagram}

The amount of devised solutions along Lambert's problem timeline is massive.
However, after spotting some common points between different solvers, it is
possible to classify them according to the free-parameter employed, the
numerical method used, the initial guess procedure or the velocity vectors
construction. Among these four, the classification based on the free-parameter
is the most useful one, as it also allows to track the evolution of a particular
solver. 

In addition to the classification presented in the following lines, an
inheritance diagram has been produced so reader can have a better understanding
about the relations between the different published solvers over time. The work
by \cite{sangra2020} has been extremely useful and expanded.

\subsection{Semi-major axis based solvers}

The first methods to solve for the Lambert's problem made use of the semi-major
axis as the free-parameter, see previously introduced Lagrange solution in
section \ref{sec:lagrange_sol}. In addition, this variable is directly involved
in the Lambert's theorem (see section \ref{sec:lamberts_theorem}).

However, the analysis made by \cite{battin1999} demonstrates that iterating over
the semi-major axis is not convenient for the case of multiple revolutions,
where a pair of conjugate orbits exist. In addition to this, the derivative of
the transfer time becomes singular when the semi-major axis $a = a_m$, being
$a_m = s / 2$ the semi-major axis of the minimum energy orbit.

The first algorithm found in literature is the one by \cite{lagrange1788}. This
author set the basis for others to improve on his method. In particular,
\cite{prussing2000} expanded the solution to the multi-revolution case and
\cite{wailliez2014} improved the convergence of the method by making use of a
Householder's root solver over a simple-semi analytical expression for the time
of flight. Finally, \cite{jiang2016} revisits the problem for both the elliptic
and hyperbolic transfer types.

However, another branch within the semi-major axis solvers was started by
\cite{thorne1995} when this author devised a series solution for the Lambert's
problem. This algorithm was improved by the same author some years later (see
\cite{thorne2004}). A convergence analysis of the series was made again by him
in \cite{thorne2015}. Notice that, even if this algorithm does not require from
a numerical solver nor initial guess, it still makes use of a free-parameter and
the velocity construction method.

\vspace{0.5cm}
\begin{figure}[h]
  \centering
  \includegraphics[scale=1.00]{static/a_solvers.pdf}
  \caption{Inheritance diagram for semi-major axis based solvers. Two branches
  exist: the one devised by Lagrange and the series-based one by Thorne.}
  \label{fig:a_solvers}
\end{figure}


\subsection{Eccentricity based solvers}

Lambert's solvers which iterative over the eccentricity $e$ of the transfer are
relatively new if compared to other solvers. The first algorithm of this type in
history seems to be the one developed by \cite{escobal1965}. However,
\cite{Battin 1999} introduced in his book an important property of the
eccentricity vector within the context of Lambert's problem: its projection
along the chord-wise direction is kept constant. 

Previous statement can be proof from the orbit equation, where $\vec{e} \cdot
\vec{r} = p - r$. If evaluated at the two known position vectors, then $\vec{e}
\cdot (\vec{r_2} - \vec{r_1}) = \norm{\vec{r_1}} - \norm{\vec{r_2}}$. Because
$\vec{c} = \vec{r_2} - \vec{r_1}$, and the norm of the vectors does not change
over time, the projection $\vec{e} \cdot \vec{c} = e_{c}$ is seen to be
constant. Some authors also refer to $e_{c}$ as $e_F$.

The first author to devise a method based on previous property was
\cite{avanzini2008}, introducing a simple algorithm which iterates over the
transverse component of the eccentricity vector $e_T$, such that $e =
\sqrt{e_c^2 + e_T^2}$. However, this algorithm was improved by \cite{he2010} who
expanded to the multi-revolution case and provided the derivative of Kepler's
equation with respect to the free-parameter. Finally, new improvements were made
by \cite{wen2014} reducing the computational cost and therefore, increasing the
overall performance.

\vspace{0.5cm}
\begin{figure}[h]
  \centering
  \includegraphics[scale=1.00]{static/ecc_solvers.pdf}
  \caption{Inheritance diagram for eccentricity based solvers. Notice that the
  branch started by Avanzini is the most modern one and the mother of modern
  solvers based on this orbit parameter.}
  \label{fig:a_solvers}
\end{figure}

\subsection{True anomaly}

Regarding solvers which iterate over the true anomaly of the transfer orbit,
only a single one has been found among the whole Lambert's problem literature:
the one devised by \cite{gunkel1960}. This algorithm also appears in the book by
\citeauthor{escobal1965}.

\vspace{0.5cm}
\begin{figure}[h]
  \centering
  \includegraphics[scale=1.00]{static/nu_solvers.pdf}
  \caption{Inheritance diagram for true anomaly based solvers. Only one solver
  is found to belong to this set of solvers as iterating over $\nu$ is not the
  most popular way of addressing the problem.}
  \label{fig:a_solvers}
\end{figure}

\subsection{Semi-latus rectum based}
\subsection{Universal formulation based}
\subsection{Kustaanheimo-Stiefel based}
\subsection{Flight path angle based}
